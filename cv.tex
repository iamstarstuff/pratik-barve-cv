%-----------------------------------------------------------------------------------------------------------------------------------------------%
%	The MIT License (MIT)
%
%	Copyright (c) 2021 Jitin Nair
%
%	Permission is hereby granted, free of charge, to any person obtaining a copy
%	of this software and associated documentation files (the "Software"), to deal
%	in the Software without restriction, including without limitation the rights
%	to use, copy, modify, merge, publish, distribute, sublicense, and/or sell
%	copies of the Software, and to permit persons to whom the Software is
%	furnished to do so, subject to the following conditions:
%	
%	THE SOFTWARE IS PROVIDED "AS IS", WITHOUT WARRANTY OF ANY KIND, EXPRESS OR
%	IMPLIED, INCLUDING BUT NOT LIMITED TO THE WARRANTIES OF MERCHANTABILITY,
%	FITNESS FOR A PARTICULAR PURPOSE AND NONINFRINGEMENT. IN NO EVENT SHALL THE
%	AUTHORS OR COPYRIGHT HOLDERS BE LIABLE FOR ANY CLAIM, DAMAGES OR OTHER
%	LIABILITY, WHETHER IN AN ACTION OF CONTRACT, TORT OR OTHERWISE, ARISING FROM,
%	OUT OF OR IN CONNECTION WITH THE SOFTWARE OR THE USE OR OTHER DEALINGS IN
%	THE SOFTWARE.
%	
%
%-----------------------------------------------------------------------------------------------------------------------------------------------%

%----------------------------------------------------------------------------------------
%	DOCUMENT DEFINITION
%----------------------------------------------------------------------------------------

% article class because we want to fully customize the page and not use a cv template
\documentclass[a4paper,12pt]{article}

%----------------------------------------------------------------------------------------
%	FONT
%----------------------------------------------------------------------------------------

% % fontspec allows you to use TTF/OTF fonts directly
% \usepackage{fontspec}
% \defaultfontfeatures{Ligatures=TeX}

% % modified for ShareLaTeX use
% \setmainfont[
% SmallCapsFont = Fontin-SmallCaps.otf,
% BoldFont = Fontin-Bold.otf,
% ItalicFont = Fontin-Italic.otf
% ]
% {Fontin.otf}

%----------------------------------------------------------------------------------------
%	PACKAGES
%----------------------------------------------------------------------------------------
\usepackage{url}
\usepackage{parskip} 	

%other packages for formatting
\RequirePackage{color}
\RequirePackage{graphicx}
\usepackage[usenames,dvipsnames]{xcolor}
\usepackage[scale=0.9]{geometry}

%tabularx environment
\usepackage{tabularx}

%for lists within experience section
\usepackage{enumitem}

% centered version of 'X' col. type
\newcolumntype{C}{>{\centering\arraybackslash}X} 

%to prevent spillover of tabular into next pages
\usepackage{supertabular}
\usepackage{tabularx}
\newlength{\fullcollw}
\setlength{\fullcollw}{0.47\textwidth}

%custom \section
\usepackage{titlesec}				
\usepackage{multicol}
\usepackage{multirow}

%CV Sections inspired by: 
%http://stefano.italians.nl/archives/26
\titleformat{\section}{\Large\scshape\raggedright}{}{0em}{}[\titlerule]
\titlespacing{\section}{0pt}{10pt}{10pt}

%for publications
\usepackage[style=authoryear,sorting=ynt, maxbibnames=2]{biblatex}

%Setup hyperref package, and colours for links
\usepackage[unicode, draft=false]{hyperref}
\definecolor{linkcolour}{rgb}{0,0.2,0.6}
\hypersetup{colorlinks,breaklinks,urlcolor=linkcolour,linkcolor=linkcolour}
\addbibresource{citations.bib}
\setlength\bibitemsep{1em}

%for social icons
\usepackage{fontawesome5}

%debug page outer frames
%\usepackage{showframe}

%----------------------------------------------------------------------------------------
%	BEGIN DOCUMENT
%----------------------------------------------------------------------------------------
\begin{document}

% non-numbered pages
\pagestyle{empty} 

%----------------------------------------------------------------------------------------
%	TITLE
%----------------------------------------------------------------------------------------

% \begin{tabularx}{\linewidth}{ @{}X X@{} }
% \huge{Your Name}\vspace{2pt} & \hfill \emoji{incoming-envelope} email@email.com \\
% \raisebox{-0.05\height}\faGithub\ username \ | \
% \raisebox{-0.00\height}\faLinkedin\ username \ | \ \raisebox{-0.05\height}\faGlobe \ mysite.com  & \hfill \emoji{calling} number
% \end{tabularx}

\begin{tabularx}{\linewidth}{@{} C @{}}
\Huge{Pratik Bhushan Barve} \\[7.5pt]
 \href{mailto:barvepratik96@gmail.com}{\raisebox{-0.05\height}\faEnvelope \ barvepratik96@gmail.com} \ $|$ \ 
\href{tel:+353894065251}{\raisebox{-0.05\height}\faMobile \ +353 8940 65251} \ $|$ \ 
\href{https://www.linkedin.com/in/pratik-barve/}{\raisebox{-0.05\height}\faLinkedin\ pratik-barve} \ $|$ \ 
\href{https://github.com/iamstarstuff}{\raisebox{-0.05\height}\faGithub\ iamstarstuff} \  
%\href{https://mysite.com}{\raisebox{-0.05\height}\faGlobe \ mysite.com} \ $|$ \ 

\end{tabularx}

%------------------------------------------------------------------------------------

%Interests/ Keywords/ Summary
\section{Summary}
Data Science graduate student at South East Technological University, Carlow, looking for an entry level Data Analyst / Data engineer / Software Engineering positions where I can apply my data visualisation and Data analysis skills to better understand the problems and seek solutions. Technology and science enthusiast, always willing to learn, explore and understand things in depth.

%This CV can also be automatically complied and published using GitHub Actions. For details, \href{https://github.com/jitinnair1/autoCV}{click here}.


%----------------------------------------------------------------------------------------
%	EDUCATION
%----------------------------------------------------------------------------------------
\section{Education}
\begin{tabularx}{\linewidth}{@{}l X@{}}	
	2022 - 2023 & \textbf{Master of Science - Data Science} \\ %\hfill \normalsize (GPA: 4.0/4.0) \\
	 (Ongoing) & \textbf{\textit{South East Technological University, Ireland}} \\
	 & → Courses : Python, R, SQL, Statistics, Visualisation in R and Pandas, Data Analytics \\
	 & \hspace{.4cm} and Algorithms, Infrastructure for Big data \\
	 & → Extra Curricular : Class rep for 2022-2023 Data Science batch. \\ 
	 & \hspace{.4cm} Clubs/Societies - Table Tennis, Everything Space\\
	\\
	2017 - 2019 & \textbf{Master of Science - Physics} \\ %\hfill (GPA: 4.0/4.0) \\ 
	 & \textbf{\textit{University of Mumbai, K. J. Somaiya College}}\\
	 & Courses : Mechanics, Mathematical Methods - Linear algebra, advance calculus, Differential equations, Electronics, Quantum Mechanics, Statistical Mechanics, Electrodynamics \\
	 \\
	2014 - 2017 &  \textbf{Bachelor of Science - Physics}\\
	& \textbf{\textit{University of Mumbai, Wilson College}}\\
	& Courses : Mechanics, Electrodynamics, Quantum Mechanics, Solid State Physics, Mathematical methods, Experimental physics, 8085 Microprocessor \\
	%2014 & Class 12th Some Board \hfill  (Grades) \\
	
	%2021 & Class 10th Some Board \hfill  (Grades) \\
\end{tabularx}

%----------------------------------------------------------------------------------------
%	SKILLS
%----------------------------------------------------------------------------------------
\section{Skills}
\begin{tabularx}{\linewidth}{@{}l X@{}}
	\textbf{Programming} &  \normalsize{Python (NumPy, Matplotlib, Pandas, Scikit-learn, Keras, Plotly, Streamlit)} \\
	& R, SQL, Git, Github\\
	&\\
	\textbf{Softwares}  &  \normalsize{\LaTeX, Microsoft Office Suite, Google Suite, Adobe Photoshop \& Lightroom}\\  
\end{tabularx}

%----------------------------------------------------------------------------------------
%PROJECTS
%----------------------------------------------------------------------------------------
\section{Projects}

\begin{tabularx}{\linewidth}{ @{}l r@{} }
	
	\large{\textbf{Maglimit}} & \hfill \href{https://github.com/iamstarstuff/maglimit}{Maglimit Github Repo} \\[3.75pt]
	\multicolumn{2}{@{}X@{}}{Published a Python package on pypi as a part of \href{https://semaphorep.github.io/codeastro/}{CodeAstro} Workshop conducted by Caltech in 2022. It is a  package to determine observability of an astronomical object using its magnitude and telescope's limiting magnitude. }  \\
	\\
	\large{\textbf{PhysicStuff}} & \hfill \href{https://github.com/iamstarstuff/PhysicStuff}{PhysicStuff Github Repo} \\[3.75pt]
	\multicolumn{2}{@{}X@{}}{Owner and author of \href{http://physicstuff.com}{PhysicStuff.com}, an informative Physics website. I have also curated a Github repo with interesting Physics and Math visualisations created in Python.}
\end{tabularx}

%----------------------------------------------------------------------------------------
% EXPERIENCE SECTIONS
%----------------------------------------------------------------------------------------

%Experience
\section{Work Experience}

\begin{tabularx}{\linewidth}{ @{}l r@{} }
\textbf{Python Tutor} & \hfill Dec 2021 - Feb 2022 \\[3.75pt]
\multicolumn{2}{@{}X@{}}{Prepared and delivered an Introductory Python course for high school students at \textit{Goldcrest International School, Vashi India.}}  \\
\end{tabularx}

\begin{tabularx}{\linewidth}{ @{}l r@{} }
\textbf{Python Tutor} & \hfill May 2021 \\[3.75pt]
\multicolumn{2}{@{}X@{}}{

        Python tutor for Computational Physics Summer school, organised by the \textit{Indian Association of Physics Teachers} for under graduate physics students.
        Prepared tutorials and assingments for active learning and 
        covered important packages like NumPy, Matplotlib, Pandas, SciPy and SymPy.
  
}
\end{tabularx}

%\begin{tabularx}{\linewidth}{ @{}l r@{} }
%\textbf{Python Tutor} & \hfill May 2021 \\[3.75pt]
%\multicolumn{2}{@{}X@{}}{
%	\begin{minipage}[t]{\linewidth}
%		\begin{itemize}[nosep,after=\strut, leftmargin=1em, itemsep=3pt]
%			\item[--] long long line of blah blah that will wrap when the table fills the column width
%			\item[--] again, long long line of blah blah that will wrap when the table fills the column width but this time even more long long line of blah blah. again, long long line of blah blah that will wrap when the table fills the column width but this time even more long long line of blah blah
%		\end{itemize}
%	\end{minipage}
%}
%\end{tabularx}

%----------------------------------------------------------------------------------------
%	RESEARCH EXPERIENCE
%----------------------------------------------------------------------------------------

\section{Research Experience}

\begin{tabularx}{\linewidth}{ @{}l X@{} }
	
	& \textbf{Jan 2021 - Sep 2022}\\
	 & \textbf{Optimising Radio Telescope Beam Pattern To Detect nearby Fast Radio Bursts} \\
	 
	 & \textbf{Guide:} Dr. Shriharsh Tendulkar (Dept. of Astronomy and Astrophysics, TIFR)\\
	 & \textit{Description:} Using the Luminosity function, we estimate the event detection rate of a radio telescope beam pattern. Comparing event rates we determine the optimal beam pattern.\\ 
	\\ 
	& \textbf{Jan 2019 - May 2019}\\
	& \textbf{Masters Thesis: Fabrication and Characterisation of Resistive plate cham- bers (RPCs) for Cosmic Muon Tracker (CMT)}\\
	& \textbf{Guide: } Dr. Satyanarayana Bheesette (Dept. of High Energy Physics, TIFR Mumbai)\\
	& \textit{Description: }Constructed and characterised a portable, fully functional cosmic muon tracker by stacking 8 RPC detector layers. \\
	\\
	& \textbf{May 2016}\\
	& \textbf{Determination of Muon Lifetime using Plastic Scintillator Detector}\\
	& \textbf{Guide: }Dr. Satyanarayana Bheesette (Dept. of High Energy Physics, TIFR Mumbai)\\ 
	& \textit{Description: }Determined lifetime of a stopped cosmic muon in a plastic scintillator detector using a simple setup of scintillator and photo multiplier tube.\\
\end{tabularx}


\section{Courses / Workshops}
	\begin{itemize}[nosep,after=\strut, leftmargin=1em, itemsep=3pt]
		\item[--] Selected for \href{https://semaphorep.github.io/codeastro/}{Code/Astro} Workshop organized by Caltech. A week long astronomy software development workshop held in 3rd week of June 2022. Published my own Python package Maglimit using knowledge gained in workshop.
		\item[--] Attended 2 consecutive \href{https://www.ztf.caltech.edu/summer-school.html}{ZTF Summer School 2021 and 2022}. Gained hands-on experience and training in data processing of ZTF and other transient survey data, using modern data science techniques such as Bayesian inference, time-series analysis, and machine learning.
		\item[--] Completed an online course - \href{https://coursera.org/share/7c2b511fa13563535f7a46a3a930f365}{"Data Driven Astronomy"} by The University of Sydney and offered on Coursera. Learned the application of SQL in big data and  machine learning in galaxy classification.
	\end{itemize}


\section{Interests / Hobbies}
Astronomy, Astrophotography, Reading, Music, Trained in Indian Classical Percussion - Tabla


%\vspace{0.3cm}
%\noindent\rule{19cm}{0.4pt}

%\textit{References provided on request}
%----------------------------------------------------------------------------------------
%	PUBLICATIONS
%----------------------------------------------------------------------------------------
%\section{Publications}
%\begin{refsection}[citations.bib]
%\nocite{*}
%\printbibliography[heading=none]
%\end{refsection}



\vfill
\center{\footnotesize References provided on request}
\center{\footnotesize Last updated: \today}

\end{document}

